\section{Hill}

	\begin{frame}
		\begin{center}
			\LARGE{\textcolor{blue}{Cifrari monoalfabetici a blocchi}}
		\end{center}
	\end{frame}

	\subsection{Intro}
	
		\begin{frame}
			\frametitle{1918-1923}		
			\begin{itemize}
				\item Il \textcolor{blue}{23 Febbraio 1918}, l'ingegnere tedesco \alert{Arthur Scherbius}, brevetta una "macchina cifrante a rotori"
				\item Nasce \alert{ENIGMA}, una delle macchine cifranti più famose della storia
				\item Crea i primi prototipi e tenta di proporla alla milizia tedesca che al momento non è interessata
				\item Decide di fondare una propria azienda per produrre ENIGMA \textcolor{blue}{per scopi commerciali}. 
			\end{itemize}
		\end{frame}

